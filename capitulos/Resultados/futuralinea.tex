\myChapter{Futuras l\'ineas de Investigaci\'on}\label{chap:web}
%\begin{flushright}{\slshape
%    Call me Ishmael.} \\ \medskip
%    --- {Herman Melville, Moby-Dick; or, The Whale}
%\end{flushright}
\minitoc\mtcskip

\clearpage
\vfill

Si bien el proyecto SIPEsCA se encuentra actualmente finalizado, se plantea seguir trabajando en los siguientes puntos a corto plazo:

\begin{itemize}
	\item Instalar m�s dispositivos de monitorizaci�n ampliando el �rea urbana cubierta, lo cual permita una malla de dispositivos mayor.
	\item Instalar dispositivos en distintas ciudades para analizar el flujo de veh�culos entre estas.
	\item Comprobar la efectividad de nuevos m�todos de predicci�n de series temporales para integrarlos en el sistema.
	\item Usar nuevas redes sociales y herramientas m�viles para ofrecer la informaci�n a los usuarios.
	\item Ampliar la aplicaci�n m�vil para servir informaci�n de manera m�s personalizadas.
\end{itemize}

En cuanto a las l�neas futuras, a medio o largo plazo, cabe destacar:

\begin{itemize}
	\item Realizar m�s an�lisis ``big data'' de los datos (no centr�ndose s�lo en pasos y trazos).
	\item Desarrollar aplicaciones m�viles para nuevos sistemas (Windows Phone, iPhone/iPad, Android L, Android Wear, Google Glass, Firefox OS).
	\item Analizar los datos con nuevas herramientas avanzadas de data-mining.
	\item Migrar el servidor local a un servidor en la nube como MS Azure o Amazon EC2 para asegurar la disponibilidad.
\end{itemize}

\vfill
\cleardoublepage
